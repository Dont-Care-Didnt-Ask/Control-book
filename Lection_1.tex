\section{Лекция 1}

Рекомендованная литература:
\begin{enumerate}
    \item Киселёв~Ю.Н. "<Оптимальное управление">, 1988
    \item Благодатских~В.И. "<Введение в оптимальное управление">, 2001
    \item Киселёв~Ю.Н., Аввакумов~С.Н., Орлов~М.В. "<Оптимальное управление. Линейная теория и приложения">, 2007 %(http://ldis-sl.cs.msu.ru/moodle_emsu/)
    \item Понтрягин~Л.С., Болтянский~В.Г., Гамкрелидзе~Р.В., Мищенко~Е.Ф. "<Математическая теория оптимальных процессов">, 1966
\end{enumerate}

Теория ОУ родилась в ходе решения задачи о моделировании боя двух самолётов (задачи уклонения/преследования), описываемой системой дифференциальных уравнений.
Задача эта до сих пор не решена ввиду её сложности.

Как правило, сложные задачи пытаются упростить.
С этой целью можно рассмотреть модель с одним самолётом. 
Однако и она для нас будет слишком сложной.
Поэтому имеет смысл начать с ещё более простой модели~--- управляемого снаряда, который нужно доставить из точки А в точку Б в наикратчайшее время.

Мы будем иметь дело с моделями, описываемыми системами обыкновенных дифференциальных уравнений.

\subsection{Описание задачи}

Рассмотрим систему дифференциальных уравнений
\begin{equation}
    \dot{x} = f(x, u)
\end{equation}
Здесь $x \in E^n $ (т.е. $x$ принадлежит пространству $\mathbb{R}^n$, где введено скалярное произведение~--- евклидову пространству).
То есть $x = \left( \begin{matrix}
    x_1 \\
    \vdots \\
    x_n 
\end{matrix} \right), \; x_i = x_i(t)$.
Аналогично, $u \in E^m, \; u = \left( \begin{matrix}
    u_1 \\
    \vdots \\
    u_m 
\end{matrix} \right), \; u_i = u_i(t)$. 
При этом их множество значений ограничено: $u_{i \min} \leqslant u_i \leqslant u_{i \max}, \; u \in U \subset E^m$.
Множество $U$ называется \textbf{областью управления}. 
Как правило, оно замкнуто.
Пример~--- $U = \left\{u = \left(\begin{matrix}
    u_1 \\
    u_2
\end{matrix}\right)\biggl| \; \cfrac{u_1^2}{a^2} + \cfrac{u_2^2}{b^2} = 1\right\}$.

Таким образом, можно переписать уравнение в виде 
\begin{equation}
    \dot{x} = f(x, u(t)) = F(x, t)
\end{equation}

Из какого класса выбираются функции $u_i(t)$?
Наиболее популярны следующие варианты:
\begin{itemize}
    \item кусочно-непрерывные функции
    \item кусочно-постоянные функции
    \item измеримые функции
\end{itemize}
Первые два часто используются в приложениях.
С первым случаем есть некоторая тонкость в определении.
В математическом анализе под кусочно-непрерывной функцией понимается следующее:
\begin{defn}
    Функция $u(t)$ называется \textbf{кусочно-непрерывной} на отрезке $[t_0, t_1]$, 
    если она непрерывна на нём всюду за исключением, может быть, конечного числа точек разрыва первого рода.
\end{defn}
В оптимальном управлении для удобства принято считать, что кусочно-непрерывная функция \textbf{непрерывна в концах отрезка} $[t_0, t_1]$. \\
Второй может соответствовать простейшим прикладным случаям, отражая, например, состояние управляющих электромагнитов: $0$~--- выключен, $1$~--- включен.
В этом случае $U = \left\{0, 1\right\}$, и функции $u_i(t)$ принимают лишь два возможных значения. \\
Третий случай нужен в основном в теоретических исследованиях.
В целом, можно выбрать и какой-нибудь другой класс, например, гладких функций.

Выбрав область управления $U$ и класс, из которого мы будем брать функции, мы определяем \textbf{класс допустимых управлений} $D_U$.
Например, $D_U = \left\{ u(t) \biggl| \begin{cases} u(t) \in U \; \forall \, t \in [t_0, t_1]\\ u(t) \in C[t_0, t_1]\end{cases} \right\}$.
%\begin{defn}
%    Класс допустимых управлений $D_U$~--- такой класс функций, что
%    \begin{compactlist}
%        \item их область значений принадлежит области управления $U$
%        \item они принадлежат некоторому из перечисленных 
%    \end{compactlist}
%\end{defn}

Для решения систем дифференциальных уравнений хотелось бы применить теоремы о существовании и единственности решений, но ввиду того,
что рассматриваемые классы достаточно широки, не всегда соблюдаются условия теорем~--- например, липшицевость функции $F(x, t) \equiv f(x, u(t))$.
Приходится либо накладывать дополнительные ограничения на условие, либо вообще по-другому вводить понятие решения дифференциального уравнения.

Конкретизируем задачу. 
Выберем некоторую фиксированную функцию $u = u(t) \in D_U$.
Пусть мы хотим, чтобы в начальный момент времени $t_0$ точка $x_0 = x(t_0)$ принадлежала некоторому множеству $M_0 \subset E^n$, 
а в конечный момент времени $t_1$ точка $x(t_1)$ принадлежала множеству $M_1 \subset E^n$.
Тогда мы должны решить, по сути, некоторую краевую задачу с условиями
\begin{equation}
    \label{lection1:boundary_val_problem}
    \begin{cases}
        \dot{x} = f(x, u) \\
        x(t_0) \in M_0 \\
        x(t_1) \in M_1
    \end{cases}
\end{equation}
Моменты времени $t_0, t_1$ могут быть как фиксированными, так и свободными.
Обычно начальный момент $t_0$ фиксирован, а $t_1$ свободно.
Если решение такой задачи существует, то пара $\bigl(u(t), x(t)\bigr)$ называется \textbf{допустимым процессом}.

Но как надо выбирать функцию $u$?
Для этого вводится \textbf{функционал качества}:
\begin{equation}
    J[u] = \int\limits_{t_0}^{t_1} f^0(x, u) \, dt
\end{equation}

Выберем наиболее "<качественную"> функцию $u(t)$~--- ту, на которой достигается минимум этого функционала 
(можно было бы искать максимум, но эти задачи эквивалентны и сводятся друг к другу умножением $f^0(x, u)$ на $-1$).
Т.е. решим задачу 
\begin{equation}
    J[u] = \int\limits_{t_0}^{t_1} f^0(x, u) \, dt \rightarrow \min\limits_{u \in D_U}
\end{equation}
где $x$~--- решение уравнения \ref{lection1:boundary_val_problem}, соответствующего функции $u$.

Т.е. полностью наша задача будет выглядеть следующим образом:
\begin{equation}
    \label{lection1:main_problem}
    \begin{cases}
        \dot{x} = f(x, u) \\
        x(t_0) \in M_0 \\
        x(t_1) \in M_1 \\
        \displaystyle J[u] = \int\limits_{t_0}^{t_1} f^0(x, u) \, dt \rightarrow \min\limits_{u \in D_U}
    \end{cases}
\end{equation}

Однако решить её в таком виде в общем случае весьма сложно.

\subsection{Линейный случай}

Разберёмся сначала с более простым случаем.
Предположим, что функция $f(x, u)$ имеет вид
\begin{equation}
    f(x, u) = Ax + Bu,
\end{equation}
где $A \in \mathbb{R}^{n \times n}, B \in \mathbb{R}^{m \times n}$.
Таким образом, в правой части нет произведений вида $x_i u_j$.
На самом деле, можно даже предположить, что функция имеет вид 
\begin{equation}
    f(x, u) = Ax + u
\end{equation}
так как всегда можно сделать замену $v = Bu$.
Чтобы не менять обозначения, предположим, что мы уже сделали эту замену и $u \in E^n$.

Кроме того, положим $f^0(x, u) \equiv 1$.
Тогда функционал качества приобретает очень простой вид: $J[u] = \int\limits_{t_0}^{t_1} 1 \, dt = t_1 - t_0.$
Таким образом, минимизируя функционал качества, мы находим такую функцию $u(t)$, 
при которой точка $x(t)$ быстрейшим образом попадает из множества $M_0$ в множество $M_1$.

Мы приходим к формулировке \textbf{линейной задачи быстродействия}:
\begin{equation}
    \label{lection1:linear_perfom_problem}
    \begin{cases}
        \dot{x} = Ax + u \\
        x(t_0) \in M_0 \\
        x(t_1) \in M_1 \\
        t_1 - t_0 \rightarrow \min\limits_{u \in D_U}
    \end{cases}
\end{equation}

\subsection{Модель тележки}

\begin{center}
    \begin{tikzpicture}
        % Axe
        \draw [->] (-8, 0) -- (8, 0) node [right] {$x$};
        \fill [black] circle[radius = 1pt] node [below] {$0$};

        % Cart
        \draw (-4, 0.5) rectangle (-2, 1.5);
        \node [above] at (-3, 1.5) {$m$};
        \draw (-3.5, 0.25) circle[radius = 0.25];
        \draw (-2.5, 0.25) circle[radius = 0.25];

        % Force
        \fill [blue!70!white] (-3, 1) circle[radius = 1pt];
        \draw [-latex, blue!70!white] (-3, 1) -- (-1, 1) node [right] {$F(t)$};
    \end{tikzpicture}
\end{center}

Рассмотрим движение тележки по прямой под действием внешней силы $F(t)$.
$x = x(t)$~--- координата.
Будем считать, что $F(t) \in [F_{\min}, F_{\max}] \; \forall \, t \in [t_0, t_1]$ (что логично~--- в реальном мире сила ограничена).
Пусть мы хотим доставить тележку в точку $x = 0$ и остановить её там.
Тогда краевые условия выглядят следующим образом:
\begin{equation}
    \begin{array}{cc}
        x(t_0) = x_0, & \dot{x}(t_0) = x_{01} \\
        x(t_1) = 0, & \dot{x}(t_1) = 0
    \end{array}
\end{equation}

Вспомним второй закон Ньютона: $m\ddot{x} = F$.
Перепишем это в виде $\ddot{x} = \cfrac{F}{m}$ и обозначим $v = \cfrac{F}{m}$.
Покажем, что уравнение $\ddot{x} = v$ можно переписать в виде, аналогичном \ref{lection1:linear_perfom_problem}.
Для этого ввёдем переменные $x_1 = x, \; x_2 = \dot{x}$.
Тогда $\dot{x_1} = x_2, \; \dot{x_2} = \ddot{x} = v$.
Для простоты считаем, что $\cfrac{F_{\min}}{m} = -1, \cfrac{F_{\max}}{m} = 1$ (можно свести задачу к этому заменой переменных).
Тогда область управления $U = [-1, 1]$.

Теперь введём матрицу $A = \left( \begin{matrix} 
    0 & 1 \\
    0 & 0 
\end{matrix} \right)$ и векторы ${\vec{x} = \left(\begin{matrix} x_1 \\ x_2 \end{matrix}\right)}, \;
                                 {\vec{u} = \left( \begin{matrix} 0 \\ u_2 \end{matrix} \right)}$.
Тогда 
\begin{equation}
    \begin{cases}
        \dot{\vec{x}} = A\vec{x} + \vec{u} \\
        \vec{x}(t_0) = \left( 
            \begin{matrix}
                x_0 \\
                x_{01}
            \end{matrix} 
        \right)\\
        \vec{x}(t_1) = \left( 
            \begin{matrix}
                0 \\
                0
            \end{matrix} 
        \right)\\
        t_1 - t_0 \rightarrow \min\limits_{u \in D_U}
    \end{cases}
\end{equation}

Таким образом, модель тележки сводится к линейной задаче быстродействия.