\documentclass[oneside,final,14pt]{extreport}

%Общий вид и оформление
\usepackage{vmargin}
\usepackage{setspace}
\usepackage{indentfirst}
\usepackage{microtype}

\setpapersize{A4}
\setmarginsrb{2cm}{1.5cm}{2cm}{1.5cm}{0pt}{0mm}{0pt}{13mm}
\linespread{1.05}
\raggedbottom
\sloppy

% Для зачеркивания текста
\usepackage[normalem]{ulem}

%Русский язык
\usepackage[nottoc,notlot,notlof]{tocbibind}
\usepackage{cmap}
\usepackage[T2A]{fontenc}
\usepackage[utf8]{inputenc}
\usepackage[english, russian]{babel}

%Математические формулы
\usepackage{amsmath}
\usepackage{amsthm}
\usepackage{amssymb}
\usepackage{nicefrac}
\allowdisplaybreaks

%Таблицы
\usepackage{adjustbox}
\usepackage{hhline}
\usepackage{multirow}
\usepackage{caption}

\newcommand{\doublerow}[2]{\begin{tabular}{@{}c@{}}#1 \\ #2\end{tabular}}

\DeclareCaptionType{mytype}[Таблица][Список] %Название таблицы без окружения table
\newenvironment{mytable}{\captionsetup{type=mytype}}{}

%Графика
\usepackage{graphicx}
\usepackage{tikz-cd}
\usetikzlibrary{calc}
\usetikzlibrary{decorations.pathmorphing}
\usepackage{pgfplots}
\pgfplotsset{compat=1.15}
\usepgfplotslibrary{fillbetween}

%Убрать отступ в начале главы
\usepackage{etoolbox}
    \makeatletter
    \patchcmd{\@makechapterhead}{\vspace*{50\p@}}{}{}{}
    \patchcmd{\@makeschapterhead}{\vspace*{50\p@}}{}{}{}
    \makeatother
    
%Библиография
\usepackage{csquotes}
\usepackage[
    backend=biber, 
    sorting=nyt,
    bibstyle=gost-authoryear,
    citestyle=gost-authoryear
]{biblatex}
\addbibresource{src/refs.bib}

%Предметный указатель
\usepackage{makeidx}

%Прочие пакеты
\usepackage{relsize}
\usepackage{enumitem}

%Запретить переносить сноски на следующую страницу
\interfootnotelinepenalty=10000

%Гиперссылки
\usepackage{hyperref}
\hypersetup{
    colorlinks,
    citecolor=black,
    filecolor=black,
    linkcolor=black,
    urlcolor=black
}

%Оформление определений, теорем, доказательств и т.п.
\renewcommand{\qedsymbol}{$\blacksquare$}

\renewenvironment{proof}{{\bfseries Доказательство.}}{\qed}


%\newtheoremstyle{custom}%    <name>
%                {\topsep}%   <space above>
%                {\topsep}%   <space below>
%                {\itshape}%  <body font>
%                {}%          <indent amount>
%                {\bfseries}% <Theorem head font>
%                {.}%         <punctuation after theorem head>
%                {\newline}%  <space after theorem head> (default .5em)
%                {}%          <Theorem head spec>
\theoremstyle{custom}
\theoremstyle{definition}
\newtheorem*{exmp}{Пример}
\newtheorem*{defn}{Определение}
\newtheorem*{symb}{Обозначение}
\newtheorem*{rmrk}{Замечание}
\newtheorem*{task}{Задача}

\theoremstyle{custom}
\newtheorem*{thm*}{Утверждение}
\newtheorem*{lem}{Лемма}
\newtheorem*{crlr}{Следствие}

\newtheoremstyle{named}{}{}{\itshape}{}{\bfseries}{.}{.5em}{\thmnote{#3}}
\theoremstyle{named}
\newtheorem*{namedthm}{Теорема}

%Компактный список
\newcommand\sbullet[1][.5]{\mathbin{\vcenter{\hbox{\scalebox{#1}{$\bullet$}}}}}
\newenvironment{compactlist}{
    \begin{list}{{$\sbullet[.75]$}}
    {\setlength\partopsep{0pt}
     \setlength\parskip{0pt}
     \setlength\parsep{0pt}
     \setlength\topsep{0pt}
     \setlength\itemsep{0pt}
    }}
    {\end{list}}

\newcommand{\CoverName}{Cover} %Нумерация обложки

% Некоторые новые команды
\newcommand{\AdjNorm}[1]{\left\lVert #1 \right\rVert}
\newcommand{\Norm}[1]{\lVert #1 \rVert}
\newcommand{\SuchThat}{\, \bigl| \,}
\newcommand{\Set}[1]{\bigl\{ #1 \bigr\}}
\newcommand{\Dist}[1]{h \bigl(#1\bigr)}